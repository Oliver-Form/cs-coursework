\documentclass[12pt,a4paper]{report}
\usepackage{setspace}
\usepackage{graphicx}
\usepackage{hyperref}
\usepackage{titlesec}

\hypersetup{
    colorlinks=true,
    linkcolor=black,
    urlcolor=blue
}

\titleformat{\chapter}{\normalfont\LARGE\bfseries}{\thechapter.}{1em}{}

\begin{document}
\pagenumbering{gobble}

%======================
%   TITLE PAGE
%======================

\begin{titlepage}
    \centering
    \vspace*{1cm}

    {\Huge \textbf{ForcePlay}}\\[0.5cm]
    {\Large A 2D Particle Physics Sandbox Built with Next.js and React}\\[1.2cm]

    \rule{\textwidth}{0.4pt}\\[0.4cm]
    {\large \textbf{OCR A-Level Computer Science Coursework}}\\
    {\large Programming Project (H446/03 or H446/04)}\\
    \rule{\textwidth}{0.4pt}\\[1.5cm]

    \begin{flushleft}
    \large
    \textbf{Candidate Name:} Oliver Form\\[0.2cm]
    \textbf{Centre Number:} \underline{\hspace{4cm}}\\[0.2cm]
    \textbf{Candidate Number:} \underline{\hspace{4cm}}\\[0.2cm]
    \textbf{Academic Year:} 2025–2026\\
    \end{flushleft}

    \vfill
    {\large \today}
\end{titlepage}

\newpage
\pagenumbering{roman}

%======================
%   TABLE OF CONTENTS
%======================

\tableofcontents
\newpage
\pagenumbering{arabic}

%====================================
%   START OF MAIN REPORT
%====================================

\chapter{Analysis}

\section{Problem Context and Computational Justification}

\subsection*{Problem Context}
\begin{itemize}
    \item ForcePlay is a 2D sandbox where users create and test point-mass particles.
    \item The app demonstrates mechanics ideas: forces, momentum, acceleration and collisions.
\end{itemize}

\subsection*{Nature of the Problem}
\begin{itemize}
    \item Particle motion is defined by position, velocity, acceleration and forces.
    \item Given the state at time $t$, physics laws determine the state at $t+\Delta t$.
    \item The simulator must handle many particles, each with its own state and interactions.
\end{itemize}

\subsection*{Computational Solvability}
\begin{itemize}
    \item Newton's laws and kinematic equations map directly to arithmetic operations.
    \item Time-stepped integration (e.g. semi-implicit Euler) updates particle states.
    \item Collision detection and response use geometric checks and vector math, which are computable.
\end{itemize}

\subsection*{Need for Computation}
\begin{itemize}
    \item Real-time updates (milliseconds) are impractical to do manually.
    \item The number of pairwise interactions grows with particle count, so code is needed to scale.
    \item Changing forces, boundaries or particle properties requires continuous recalculation.
\end{itemize}

\subsection*{Why a Programmatic Simulation}
\begin{itemize}
    \item Software gives consistent, reproducible results from clear rules.
    \item The UI can update visuals instantly in response to user input.
    \item Controls like pause, step and reset need an accurate internal state, which software provides.
    \item Real-time rendering makes experiments easy to observe and repeat.
\end{itemize}

\section{Stakeholders}

\subsection*{A-level Students}
\begin{itemize}
    \item Main users who learn mechanics by experimenting.
    \item Need adjustable properties, collision setups, and pause/step controls.
\end{itemize}

\subsection*{Teachers}
\begin{itemize}
    \item Use the app for demonstrations and lessons.
    \item Need clear visuals, presets and easy classroom use.
\end{itemize}

\subsection*{Hobbyists}
\begin{itemize}
    \item Interested in exploring physics informally.
    \item Prefer simple controls and the ability to save/load scenes.
\end{itemize}

\subsection*{Developer / Tester}
\begin{itemize}
    \item Implements and tests the project.
    \item Needs reproducible simulations and easy debugging.
\end{itemize}

\section{Research Into Existing Solutions}

Checked tools like Algodoo, PhET, GeoGebra and Box2D demos. Key takeaways:
\begin{itemize}
    \item Real-time feedback is important.
    \item Simple particle visuals help learning.
    \item Allowing users to change mass, velocity and elasticity encourages experiments.
    \item Keep the interface focused and not overloaded with features.
\end{itemize}

ForcePlay aims to be component-based and minimal so it fits A-level topics and classroom use.

\section{Requirements Specification}

\subsubsection{Functional Requirements}
\begin{itemize}
    \item Spawn Particles: Users can create particles with position, mass and velocity.
    \item Apply Forces: Gravity and user-applied forces affect particles.
    \item Collision Handling: Detect and respond to collisions using impulse-based methods.
    \item Adjustable Environment: Users can change gravity, friction and boundaries.
    \item UI Controls: Pause, reset, step and speed controls are provided.
    \item Save/Load: Optional scene saving/loading for reusing setups.
    \item Performance: Keep the simulation smooth with multiple particles.
\end{itemize}

\subsubsection{Non-Functional Requirements}
\begin{itemize}
    \item Usability: Simple, clear interface with immediate feedback.
    \item Efficiency: Target 60 FPS with a reasonable number of particles.
    \item Reliability: Handle normal use without crashes.
    \item Browser Compatibility: Works in modern browsers (Chrome, Firefox).
    \item Expandability: Design allows adding features later (e.g. new forces).
\end{itemize}

\section{Hardware and Software Requirements}

\begin{itemize}
    \item Modern browser that supports React and canvas/WebGL.
    \item JavaScript runtime for the simulation logic.
    \item Next.js development stack for building the app.
    \item Decent CPU for real-time updates with many particles.
\end{itemize}

\section{Measurable Success Criteria}

\begin{itemize}
    \item Simulation runs at 60 updates per second with up to 200 particles.
    \item Collisions are correct in the majority of tested cases (aim >95%).
    \item UI responds within ~100 ms to user actions.
    \item Users can run at least three different scenarios without errors.
    \item Functional requirements from earlier are met and demonstrable.
\end{itemize}

\section{Limitations}
\begin{itemize}
    \item Point masses only (no rigid bodies or shapes).
    \item Simplified collision model (instant impulses, no deformation).
    \item No rotation — only translational motion is simulated.
    \item Numerical integration has truncation error; small timesteps may be needed.
    \item JavaScript performance limits maximum particle counts.
    \item 2D only — no 3D simulations.
    \item React + Canvas has overhead; very large scenes may need WebGL or worker-based physics.
\end{itemize}

\end{document}
